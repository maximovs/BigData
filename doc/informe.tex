\documentclass[a4paper,10pt]{article}

\usepackage[utf8]{inputenc}
\usepackage{t1enc}

\usepackage[utf8]{inputenc}
\usepackage{t1enc}
\usepackage[spanish]{babel}
\usepackage[pdftex,usenames,dvipsnames]{color}
\usepackage[pdftex]{graphicx}
\usepackage{enumerate}
\usepackage{amsmath}
\usepackage{amsfonts}
\usepackage{amssymb}
\usepackage[table]{xcolor}
\usepackage[small,bf]{caption}
\usepackage{float}
\usepackage{subfig}
\usepackage{listings}
\usepackage{bm}
\usepackage{times}

\begin{document}
\setcounter{secnumdepth}{5}
\setcounter{tocdepth}{5}

\renewcommand{\lstlistingname}{C\'odigo Fuente}
\lstloadlanguages{Octave}
\lstdefinelanguage{MyOctave}[]{Octave}{
        deletekeywords={beta,det},
        morekeywords={repmat}
}
\lstset{
        language=MyOctave,
        stringstyle=\ttfamily,
        showstringspaces = false,
        basicstyle=\footnotesize\ttfamily,
        commentstyle=\color{gray},
        keywordstyle=\bfseries,
        numbers=left,
        numberstyle=\ttfamily\footnotesize,
        stepnumber=1,
        framexleftmargin=0.20cm,
        numbersep=0.37cm,
        backgroundcolor=\color{white},
        showspaces=false,
        showtabs=false,
        frame=l,
        tabsize=4,
        captionpos=b,
        breaklines=true,
        breakatwhitespace=false,
        mathescape=true
}

%%%%%%%%%%%%%%%%%%%%%%%%%%%%%%%%%%
%%%%%%%% begin TITLE PAGE %%%%%%%%
%%%%%%%%%%%%%%%%%%%%%%%%%%%%%%%%%%
\begin{titlepage}
        \vfill
        \thispagestyle{empty}
        \begin{center}
                \includegraphics{./images/itba_logo.png}
                \vfill
                \Huge{Big Data}\\
                \vspace{1cm}
                \Huge{Trabajo Pr\'actico 1}\\
        \end{center}
        \vfill
        \large{
        \begin{tabular}{lcr}
                Crespo, Alvaro && 50758 \\
                Petit, Alejandro && XXXXX \\
                Susnisky, Dario && 50592 \\
                Videla, Máximo && 51071\\
        \end{tabular}
}
        \vspace{2cm}
        \begin{center}
                \large{07 de octubre de 2013}\\
        \end{center}
\end{titlepage}
\newpage

\setcounter{page}{1}

% \tableofcontents
% \newpage
\section{Problemas encontrados}

Al correr inicialmente los jobs de map-reduce que utilizaban tablas de HBase, tuvimos errores por no setear correctamente la configuración de Zookeeper. Estos errores
pudieron solucionarse configurando correctamente las propiedades $HBASE_CONFIGURATION_ZOOKEEPER_QUORUM$ y $HBASE_CONFIGURATION_ZOOKEEPER_CLIENTPORT$.\\

Al momento de implementar joins con los archivos CSV, nos encontramos con el problema de como hacer que los archivos estén disponibles en cada mapper. La clase
\textit{Distributed Cache} nos sirvió para justamente sobrepasar esta dificultad.\\

Al correr Pig en forma map-reduce con hadoop pseudo-distribuido, tuvimos problemas en un momento ya que hadoop recordaba la IP externa de la computadora y,
por lo tanto, no encontraba el reducer.\\

En varios momentos, tanto al implementar las métricas con MapReduce en Java y con Hive, nos encontramos con irritantes errores de formato en algunos archivos CSV. En particular,
en los archivos \textit{airports.csv} y \textit{carriers.csv}, todos los campos estabán entre comillas dobles (``) y esto producía algunos errores al hacer joins. Es por esto que
previo a hacer los joins tuvimos que efectuar algunas moficiaciones en los datos, obviamente sin modificar los archivos del HDFS. Simplemente es eliminaron las comillas dobles
previo a la utilización de los datos de estos archivos.\\

Respecto a Hive, encontramos ciertos problemas al parametrizar los \textit{scripts}. Esto trajo particulares complicaciones al implementar la métrica 6, ya que en dicha métrica es posible recibir o no un parametro. En nuestros \textit{scripts} explotamos el hecho de que Hive permite el uso de variables del sistema mediante la linea de comandos (con el uso de -hiveconf). Una vez establecidas estas variables, Hive reemplaza las apariciones de estas variables por su valor real de forma similar a como funciona una macro. Sin embargo, en caso de no existir alguna variable, Hive las toma como texto plano. Para poder manejar estas situaciones, creamos una UDF que trata de discriminar si la variable existe, abusando del conocimiento que toda variable no existente contenera \"hiveconf\" en su texto.

Hubo un segundo problema con Hive, a la hora de imprimir las salidas a archivos. La separación de los campos y las tuplas no son cómodos a la hora de leerlos, haciendo las salidas poco legibles. Esto podría solucionarse rápidamente con un \textit{script} de \textit{Bash} o ejecutando busquedas y reemplazos en editores con esta capacidad (como por ejemplo, \textit{vim}).

\section{Decisiones de implementación}

Para las métricas implementadas con MapReduce en Java, decidimos implementar, en todos los casos, \textit{Broadcast Join} al hacer joins por 2 razones principales: en todos los casos,
uno de los \textit{dataset} siempre se podía asumir pequeño (3000 aeropuertos, 1500 aerolíneas y 5000 aviones), y además es la opción más fácil de implementar (en algunos casos se
utilizaban tablas de \textit{HBase} y en otros se aprovechaba el \textit{Distributed Cache} de Hadoop para distribuir los archivos CSV).\\

Decidimos implementar 2 simples métricas extra, como son la cantidad de vuelos totales de cada aerolínea, y la proporción de vuelos cancelados sobre el total de vuelos para cada
aerolínea. Otras métricas interesantes hubieran sido agregarle a estas métricas extra la posibilidad de discriminar por año además de por aerolínea, ya sea la cantidad de vuelos
o la proporción de vuelos cancelados.

\section{Instrucciones para ejecutar las métricas}

    \subsection{Map Reduce}
        \subsubsection{Métrica 1 - Promedio de demora de despegue por mes por estado}
            \scriptsize{hadoop jar bigdata-tp1-jar-with-dependencies.jar -inPath 'input\_path' -outPath 'output\_path' -avgTakeOffDelay}
        \subsubsection{Métrica 2 - Vuelos Cancelados por aerolínea}
            \scriptsize{hadoop jar bigdata-tp1-jar-with-dependencies.jar -inPath 'input\_path' -outPath 'output\_path' -cancelledFlights -carriersPath 'carriers\_path'}
        \subsubsection{Métrica 3 - Millas voladas por aerolínea por año}
            \scriptsize{hadoop jar bigdata-tp1-jar-with-dependencies.jar -inPath 'input\_path' -outPath 'output\_path' -milesFlown -carriersPath 'carriers\_path'}
        \subsubsection{Métrica 4 - Horas de vuelo por fabricante}
            \scriptsize{hadoop jar bigdata-tp1-jar-with-dependencies.jar -inPath 'input\_path' -outPath 'output\_path' -flightHours -manufacturer 'target\_manufacturer\_name'}
       \subsubsection{Métrica 13(OPCIONAL) - Cantidad de vuelos por aerolínea}
            \scriptsize{hadoop jar bigdata-tp1-jar-with-dependencies.jar -inPath 'input\_path' -outPath 'output\_path' -flightCount -carriersPath 'carriers\_path'}
        \subsubsection{Métrica 14(OPCIONAL) - Proporción de vuelos cancelados por aerolínea}
            \scriptsize{hadoop jar bigdata-tp1-jar-with-dependencies.jar -inPath 'input\_path' -outPath 'output\_path' -propCancelledFLights -carriersPath 'carriers\_path'}

    \subsection{Hive}
        \subsubsection{Métrica 5 - Top 5 Aeropuertos con demora de despegue por año}
        \scriptsize{hive -S -f metric5-depDelayTop5.sql -hiveconf flightsPath='input\_flights\_path' airportsPath='input\_airports\_path' output='output\_path'}
        \subsubsection{Métrica 6 - Imprevistos 2005}
        \scriptsize{hive -S -f metric6-2005FlightStats.sql -hiveconf flightsPath='input\_flights\_path' [airport='airport\_IATA'] output='output\_path'}
        \subsubsection{Métrica 7 - Top 5 Aeropuertos con mayor promedio de demoras}
        \scriptsize{hive -S -f metric7.sql -hiveconf flightsPath='input\_flights\_path' airportsPath='input\_airports\_path' output='output\_path'}
        \subsubsection{Métrica 8 - Huracanes con más cancelaciones}
        \scriptsize{hive -S -f metric8.sql -hiveconf flightsPath='input\_flights\_path' output='output\_path'}

    \subsection{Pig}
        \subsubsection{Métrica 9 - Rutas más voladas por año}
            \scriptsize{pig -param flights=FLIGHTS\_PATH -param airports=AIRPORTS\_PATH -param output=OUTPUT\_PATH ej9.pig}
        \subsubsection{Métrica 10 - Cantidad de vuelos cancelados y no cancelados en septiembre de 2011}
            \scriptsize{pig -param flights=FLIGHTS\_PATH -param output=OUTPUT\_PATH ej10.pig}
        \subsubsection{Métrica 11 - Hora partida del último vuelo del 9/11 para cada aeropuerto}
            \scriptsize{pig -param flights=FLIGHTS\_PATH -param output=OUTPUT\_PATH ej11.pig}
        \subsubsection{Métrica 12 - Hora partida del último vuelo del 9/11 para cada aeropuerto}
            \scriptsize{pig -param flights=FLIGHTS\_PATH -param output=OUTPUT\_PATH ej11.pig}
        \subsubsection{Métrica 13 - Promedio diario de demora de despegue en el año 2001 }
            \scriptsize{pig -param flights=FLIGHTS\_PATH -param output=OUTPUT\_PATH ej12.pig}

\section{Formato de los resultados de las métricas}

    \subsection{Map Reduce}
        \subsubsection{Métrica 1 - Promedio de demora de despegue por mes por estado}
            Los resultados de esta métrica tienen el siguiente formato:\\
            \begin{center}
                ESTADO-MES PROMEDIO \\
            \end{center}
            donde el estado se representa por dos letrás mayúsculas (su código postal) y el mes escrito en letras y en ingleś.

            Un ejemplo sería\\
            \begin{center}
                WY-September    10.518987341772151\\
            \end{center}

        \subsubsection{Métrica 2 - Vuelos Cancelados por aerolínea}
            Los resultados de esta métrica tienen el siguiente formato:\\
            \begin{center}
                AEROLINEA CANTIDAD\\
            \end{center}

            Un ejemplo sería\\
            \begin{center}
               United Air Lines Inc.   34\\
            \end{center}

        \subsubsection{Métrica 3 - Millas voladas por aerolínea por año}
            Los resultados de esta métrica tienen el siguiente formato:\\
            \begin{center}
                AEROLINEA-AÑO CANTIDAD\\
            \end{center}

            Un ejemplo sería\\
            \begin{center}
               Delta Air Lines Inc.-1987   1171792\\
            \end{center}

        \subsubsection{Métrica 4 - Horas de vuelo por fabricante}
            Los resultados de esta métrica tienen el siguiente formato:\\
            \begin{center}
                NRO\_AVION CANTIDAD\_HORAS
            \end{center}
            Cabe destacar que la cantidad de horas se encuentra en decimal ya que no se efectuaron redondeos.

            Un ejemplo sería\\
            \begin{center}
                N997AT  191.93333333333334\\
            \end{center}

        \subsubsection{Métrica 13(OPCIONAL) - Cantidad de vuelos por aerolínea}
            Los resultados de esta métrica tienen el siguiente formato:\\
            \begin{center}
                AEROLINEA CANTIDAD\\
            \end{center}

            Un ejemplo sería\\
            \begin{center}
               United Air Lines Inc.   324\\
            \end{center}

         \subsubsection{Métrica 14(OPCIONAL) - Proporción de vuelos cancelados por aerolínea}
            Los resultados de esta métrica tienen el siguiente formato:\\
            \begin{center}
                AEROLINEA PROPORCION\\
            \end{center}

            Un ejemplo sería\\
            \begin{center}
               Delta Air Lines Inc.    0.008937960042060988\\
            \end{center}

        \subsection{Hive}
        El output de Hive se puede ver en un archivo en el directorio especificado al ejecutar el \textit{script}. Como fue mencionado en los problemas encontrados, la separación entre los campos se da por el caracter \"^A\" mientras que las tuplas están separados por \"\\N\"..
        A continuación se detalla el órden de los campos para cada una de las métricas.
        \subsubsection{Métrica 5 - Top 5 Aeropuertos con demora de despegue por año}
        Año - Puesto - Aeropuerto - Horas totales de demora
        \subsubsection{Métrica 6 - Imprevistos 2005}
        Fecha - Cantidad de vuelos demorados - Suma de horas de demora - Cantidad de vuelos demorados - Cantidad de vuelos desviados - Cantidad de vuelos cancelados por mal clima
        \subsubsection{Métrica 7 - Top 5 Aeropuertos con mayor promedio de demoras}
        Aeropuerto - Puesto - Promedio de vuelos demorados por día
        \subsubsection{Métrica 8 - Huracanes con más cancelaciones}
        Huracan - Fecha - Cantidad de vuelos cancelados




    \subsection{Pig}
       \subsubsection{Métrica 9 - Las rutas más voladas}
            Los resultados de esta métrica tienen el siguiente formato:\\
            \begin{center}
                AÑO ORIGEN DESTINO CANTIDAD\_VUELOS
            \end{center}
            Se ordenan por año y, dentro de cada año, por cantidad de vuelos. Sólo se toman en cuenta los primeros diez de cada año.

            Un ejemplo sería\\
            \begin{center}
                2000  Warsaw Municipal  Salem Memorial  123456\\
            \end{center}

       \subsubsection{Métrica 10 - Cantidad de vuelos cancelados y no cancelados en septiembre de 2011}
            Los resultados de esta métrica tienen el siguiente formato:\\
            \begin{center}
                FECHA CANTIDAD\_VUELOS  CANTIDAD\_CANCELADOS
            \end{center}
            Cabe destacar que no se tiene en cuenta los casos de los vuelos reprogramados, es decir, un vuelo cancelado que se efectúa otro día en el mismo mes, suma a ambos.

            Un ejemplo sería\\
            \begin{center}
                29/9/2001  112  14\\
            \end{center}

       \subsubsection{Métrica 11 - Hora partida del último vuelo del 9/11 para cada aeropuerto}
            Los resultados de esta métrica tienen el siguiente formato:\\
            \begin{center}
                ORIGEN HORARIO
            \end{center}
            Cabe destacar que las horas se muestran en el formato h:m pero no se agrega un 0 adelante de los minutos si son menores a 10, es decir,
            las seis y cinco de la tarde se representan de la siguiente manera: 6:5. Otro comentario sería que se listan todos los aeropuertos, incluso
            en los que no se realizó ningún vuelo, estos se indican dejando la hora en blanco.

            Un ejemplo sería\\
            \begin{center}
                SYR  8:46\\
            \end{center}

        \subsubsection{Métrica 12 - Promedio diario de demora de despegue en el año 2001}
            Los resultados de esta métrica tienen el siguiente formato:\\
            \begin{center}
                AÑO-MES-DIA PROMEDIO
            \end{center}
            Cabe destacar que la cantidad de horas se encuentra en decimal ya que no se efectuaron redondeos.

            Un ejemplo sería\\
            \begin{center}
                2001-12-31  0.298048048048048\\
            \end{center}
\end{document}
