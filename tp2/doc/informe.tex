\documentclass[a4paper,10pt]{article}

\usepackage[utf8]{inputenc}
\usepackage{t1enc}

\usepackage[utf8]{inputenc}
\usepackage{t1enc}
\usepackage[spanish]{babel}
\usepackage[pdftex,usenames,dvipsnames]{color}
\usepackage[pdftex]{graphicx}
\usepackage{enumerate}
\usepackage{amsmath}
\usepackage{amsfonts}
\usepackage{amssymb}
\usepackage[table]{xcolor}
\usepackage[small,bf]{caption}
\usepackage{float}
\usepackage{subfig}
\usepackage{listings}
\usepackage{bm}
\usepackage{times}

\begin{document}
\setcounter{secnumdepth}{5}
\setcounter{tocdepth}{5}

\renewcommand{\lstlistingname}{C\'odigo Fuente}
\lstloadlanguages{Octave}
\lstdefinelanguage{MyOctave}[]{Octave}{
        deletekeywords={beta,det},
        morekeywords={repmat}
}
\lstset{
        language=MyOctave,
        stringstyle=\ttfamily,
        showstringspaces = false,
        basicstyle=\footnotesize\ttfamily,
        commentstyle=\color{gray},
        keywordstyle=\bfseries,
        numbers=left,
        numberstyle=\ttfamily\footnotesize,
        stepnumber=1,
        framexleftmargin=0.20cm,
        numbersep=0.37cm,
        backgroundcolor=\color{white},
        showspaces=false,
        showtabs=false,
        frame=l,
        tabsize=4,
        captionpos=b,
        breaklines=true,
        breakatwhitespace=false,
        mathescape=true
}

%%%%%%%%%%%%%%%%%%%%%%%%%%%%%%%%%%
%%%%%%%% begin TITLE PAGE %%%%%%%%
%%%%%%%%%%%%%%%%%%%%%%%%%%%%%%%%%%
\begin{titlepage}
        \vfill
        \thispagestyle{empty}
        \begin{center}
                \includegraphics{./images/itba_logo.png}
                \vfill
                \Huge{Big Data}\\
                \vspace{1cm}
                \Huge{Trabajo Pr\'actico 2}\\
        \end{center}
        \vfill
        \large{
        \begin{tabular}{lcr}
                Crespo, Alvaro && 50758 \\
                Petit, Alejandro && 48308\\
                Susnisky, Dario && 50592 \\
                Videla, Máximo && 51071\\
        \end{tabular}
}
        \vspace{2cm}
        \begin{center}
                \large{18 de noviembre de 2013}\\
        \end{center}
\end{titlepage}
\newpage

\setcounter{page}{1}

% \tableofcontents
% \newpage
\section{Problemas encontrados}

\section{Agente de Flume}
El principal problema, o desafío, de la implementación del agente de Flume fue la necesidad de implementar un \textit{Sink} custom, para interactuar con la cola
de \textbf{ActiveMQ}, ya el tanto para el \textit{Souce} como para los \textit{Channels}, utilizamos implementaciones como \text{Spooling Directory Source} o \textit{Memory Channel},
que ya vienen provistas por Flume.

Un problema, o una lección, que tuvimos con Flume fue que si bien es posible que un \textit{source} se conecte a varios \textit{channels}, no sucede lo mismo con los \textit{sinks}.
Un \textit{sink} puede estar conectado a un y sólo un \textit{channel}. Además, al configurar el \textit{source} de nuestro agente con varios \textit{channels}, lo que 
se denomina \textit{Fan Out Flow}, nos topamos con las
diferentes formas en que un source puede estar configurado para manejar conexiones a varios \textit{channels}, o la ``política de Fan Out''. Existen dos formas o políticas para esto:
replicar o multiplexar. En el primer caso, el \textit{source} le manda el evento generado a todos los \textit{channels} conectados, mientrás que en el segundo, el evento se envía
a un subconjunto de ellos. En nuestro caso, nos topamos con un error al configurar varios \textit{channels} ya que, al parecer, el default en estos casos es multiplexar, lo cual no 
era lo que necesitabamos.

\section{Topología de Storm}

\section{Decisiones de implementación}

\section{Agente de Flume}
Con respecto al agente de Flume se implementó un simple agente con un \textit{source}, 3 \textit{sinks} y 3 \textit{channels}, uno que vincule cada \textit{sink} con el 
\textit{source}. 

Para el \textit{source} se utilizó un \textit{Spooling Directory Source}, que viene dado en el paquete de Flume, que simplemente se queda observando un 
determinado directorio y cuando se agregan nuevos archivos, genera un evento nuevo por cada línea. En nuestro caso, cada línea es un objeto JSON con la información de un tweet.

La decisión de tener varios \textit{sinks} vino por diferentes razones. En primer lugar, se utilizaron el \textit{Logger Sink} y el \textit{File Roll Sink}, que dirigen el output
del agente de Flume a consola (mediante logs de debugging de \textit{Log4j}) y a archivos de texto en un directorio de salida, respectivamente. Esto tenía el objetivo principal de
familiarizarnos con la nueva tecnología y a la vez ver el funcionamiento del \text{Spooling Directory Source} para nuestras necesidades. Pero luego también decidimos continuar 
utilizandolas para testear y verificar el correcto funcionamiento del \textit{ActiveMQ Sink} que implementamos.

\section{Topología de Storm}


\section{Instrucciones para ejecutar la métrica}

\small
\section{Formato de los resultado de la métrica}

Los resultados de la métrica implementada se almacenan en una tabla de MYSQL, conforme a los requerimientos de la cátedra. El esquema que definimos para esta tabla consiste 
simplemente de los campos:

\begin{itemize}
    \item GroupId: El nombre del grupo o conjunto de palabras de interés
    \item Ammount: La cantidad de menciones de palabras relacionadas con el grupo
\end{itemize}


\end{document}
